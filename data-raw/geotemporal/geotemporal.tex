\documentclass[b5paper,10pt]{scrartcl}
\usepackage{geometry}
\usepackage{lmodern}
\usepackage{microtype}
\usepackage{float}
\usepackage{booktabs}
\usepackage{tabularx}
\newcolumntype{R}{>{\raggedleft\arraybackslash $}X<$}

%$

\title{Geotemporal Self-joining}
\author{Hugh Parsonage}

\begin{document}
\maketitle

\section{Data}
We start with a dataset of geotemporal points:

\begin{table}[H]
\caption{Original data}
\begin{tabularx}{0.565\linewidth}{lRRR}
\toprule
\textbf{Idx} & \textbf{lat} & \textbf{lon} & \textbf{time} \\
\midrule
1            & -37.001      & 150.001      & 1639582230 \\
2            & -37.001      & 150.001      & 1639582230 \\
3            & -37.001      & 150.001      & 1639584611 \\
4            & -37.001      & 150.003      & 1639582230 \\
$\vdots$     & $\vdots$     & $\vdots$     & $\vdots$ \\
$N$          & -35.999      & 155.953      & 1638840278 \\
\bottomrule
\end{tabularx}
\end{table}

Then we seek the unique self-join of the geospatial coordinates 
and the self-join of time.


\end{document}

